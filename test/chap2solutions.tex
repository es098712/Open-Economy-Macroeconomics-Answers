\documentclass{article}
\usepackage{amsmath, amssymb}

\begin{document}

\title{Solutions to the Exercises of Chapter 2: An Open Endowment Economy}
\date{}
\maketitle

\section*{Exercise 2.1 (Consumption Innovations)}
In the economy with AR(1) endowment shocks studied in section 2.2, we found that 
\[
E_t c_{t+1} = c_t
\]
which means that 
\[
c_{t+1} = c_t + \mu_{t+1},
\]
where \(\mu_{t+1}\) is a white noise process that is unforecastable given information available in \(t\). 
Derive the innovation \(\mu_{t+1}\) as a function of \(r\), \(\rho\), and \(\epsilon_{t+1}\).

\textbf{Answer:} 
\[
\mu_{t+1} = \frac{r}{1+r} \rho \epsilon_{t+1}
\]

\section*{Exercise 2.2 (An Economy with Endogenous Labor Supply)}
Consider a small open economy populated by a large number of households with preferences 
described by the utility function
\[
E_0 \sum_{t=0}^{\infty} \beta^t U(c_t, h_t),
\]
where \( U \) is a period utility function given by
\[
U(c_t, h_t) = -\frac{1}{2} \left[ (\bar{c} - c_t)^2 + h_t^2 \right],
\]
where \(\bar{c} > 0\) is a satiation point. 

The household’s budget constraint is given by
\[
d_t = (1 + r) d_{t-1} + c_t - y_t,
\]
where \( d_t \) denotes real debt acquired in period \( t \) and due in period \( t+1 \), and \( r > 0 \) denotes the world interest rate.

\end{document}
