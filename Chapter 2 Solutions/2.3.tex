\documentclass{article}
\usepackage{amsmath, amssymb, amsthm}

\begin{document}

\section*{Exercise 2.3 (An Open Economy with Habit Formation, I)}

\textit{Consider a two-period small open economy populated by a large number of identical households with preferences specified by the utility function}
\begin{equation}
    \ln c_1 + \ln (c_2 - x),
\end{equation}
where $c_1$ and $c_2$ denote, respectively, consumption in periods 1 and 2. Households are endowed with $y > 0$ units of goods each period and are born in period 1 with no assets or debts. In period 1, households can borrow or lend at a zero interest rate. Derive the equilibrium level of consumption and the trade balance under the following three formulations:

\begin{enumerate}
    \item $x = 0$ (no habits).
    \item $x = 0.5 c_1$ (internal habit formation).
    \item $x = 0.5 \bar{c_1}$, where $\bar{c_1}$ denotes the economy’s per capita level of consumption in period 1 (external habit formation).
\end{enumerate}

Compare economies (1) and (2) and provide intuition. Similarly, compare economies (2) and (3) and provide intuition.

\subsection*{Answer:}

\textbf{1. The intertemporal budget constraint is}
\begin{equation}
    c_2 = 2y - c_1.
\end{equation}

\textbf{In the economy without habits, the optimality condition is}
\begin{equation}
    \frac{1}{c_1} = \frac{1}{2y - c_1},
\end{equation}
which yields
\begin{equation}
    \boxed{c_1 = y}
\end{equation}

\textbf{2. With internal habits, the household’s problem is to pick} $c_1$ \textbf{to maximize} $\ln c_1 + \ln (2y - 1.5c_1)$. \textbf{The optimality condition is}
\begin{equation}
    \frac{1}{c_1} = \frac{1.5}{2y - 1.5c_1},
\end{equation}
which yields
\begin{equation}
    \boxed{c_1 = \frac{2}{3} y}
\end{equation}

\textbf{3. With external habits, the household’s problem is to pick} $c_1$ \textbf{to maximize} $\ln c_1 + \ln (2y - c_1 - 0.5 \bar{c_1})$. \textbf{The optimality condition is}
\begin{equation}
    \frac{1}{c_1} = \frac{1}{2y - c_1 - 0.5\bar{c_1}}.
\end{equation}

\textbf{In equilibrium,} $c_1 = \bar{c_1}$. \textbf{Using this expression to eliminate} $\bar{c_1}$, \textbf{we obtain}
\begin{equation}
    \boxed{c_1 = \frac{4}{5} y}
\end{equation}

\textbf{Comparison of no habits with internal habits:} Internal habits deliver less consumption in period 1 because households internalize that the more they consume in period 1, the less happy they are in period 2. 

\textbf{Comparison of internal and external habits:} Again, with internal habits, households internalize the fact that period-1 consumption makes them unhappy in period 2. This internalization is absent under external habits, so households consume more in period 1 under the latter formulation. 

It is of interest to note that period-1 consumption is lower under external habits than under no habits. This is because under external habits, when $c_2 - c_1 = 0.5 c_1$, the marginal utility of consumption is higher in period 2 than in period 1, tilting consumption toward the future.

\end{document}
