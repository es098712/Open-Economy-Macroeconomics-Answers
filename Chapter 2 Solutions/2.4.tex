\documentclass{article}
\usepackage{amsmath, amssymb}

\begin{document}

\section*{Exercise 2.4 (An Open Economy With Habit Formation, II)} \textit{Section 2.2 characterizes the equilibrium dynamics of a small open economy with time separable preferences driven by stationary endowment shocks. It shows that a positive endowment shock induces an improvement in the trade balance on impact. This prediction, we argued, was at odds with the empirical evidence presented in Chapter 1. Consider now a variant of the aforementioned model economy in which the representative consumer has time nonseparable preferences described by the utility function}

\[
- \frac{1}{2} E_t \sum_{j=0}^{\infty} \beta^j [ c_{t+j} - \alpha \bar{c}_{t+j-1} - \bar{c} ]^2, \quad t \geq 0,
\]

\textit{where \( c_t \) denotes consumption in period \( t \), \( \bar{c}_t \) denotes the cross-sectional average level of consumption in period \( t \), \( E_t \) denotes the mathematical expectations operator conditional on information available in period \( t \), and \( \beta \in (0,1) \), \( \alpha \in (-1,1) \), and \( \bar{c} > 0 \) are parameters. The case \( \alpha = 0 \) corresponds to time separable preferences, which is studied in the main text. Households take as given the evolution of \( \bar{c}_t \). Households can borrow and lend in international financial markets at the constant interest rate \( r \). For simplicity, assume that \( (1+r)\beta \) equals unity. In addition, each period \( t = 0,1, \dots \) the household is endowed with an exogenous and stochastic amount of goods \( y_t \). The endowment stream follows an AR(1) process of the form}

\[
y_{t+1} = \rho y_t + \epsilon_{t+1}.
\]

where $\rho \in [0,1)$ is a parameter and $\varepsilon_t$ is a mean-zero i.i.d. shock. Households are subject to the no-Ponzi-game constraint

\[
\lim_{j \to \infty} \mathbb{E}_t \left( \frac{d_{t+j}}{(1 + r)^j} \right) \leq 0,
\]

where $d_t$ denotes the representative household’s net debt position at date $t$. At the beginning of period 0, the household inherits a stock of debt equal to $d_{-1}$.

\begin{enumerate}
    \item Derive the initial equilibrium response of consumption to a unit endowment shock in period 0.
    
    \item Discuss conditions (i.e., parameter restrictions), if any, under which a positive output shock can lead to a deterioration of the trade balance.
\end{enumerate}

\textbf{Answer:}

\begin{enumerate}
    \item The equilibrium conditions of this model are:

    \begin{align}
    x_t &= \mathbb{E}_t x_{t+1} \quad \text{(2.1EX)} \\
    x_t &\equiv c_t - \alpha c_{t-1} \quad \text{(2.2EX)} \\
    d_t &= (1 + r)d_{t-1} + c_t - y_t \quad \text{(2.3EX)} \\
    \lim_{j \to \infty} \mathbb{E}_t \left( \frac{d_{t+j}}{(1 + r)^j} \right) &= 0 \quad \text{(2.4EX)}
    \end{align}

    From (2.1EX) and (2.2EX) we get

    \[
    \mathbb{E}_t c_{t+j} = \alpha^{j+1} c_{t-1} + \frac{1 - \alpha^{j+1}}{1 - \alpha} x_t
    \]

    It follows that

    \[
    \mathbb{E}_t \left( \sum_{j=0}^\infty \beta^j c_{t+j} \right) = \frac{\alpha}{1 - \alpha \beta} c_{t-1} + \left[ \frac{1}{1 - \beta} - \frac{\alpha}{1 - \alpha \beta} \right] \frac{x_t}{1 - \alpha} = \frac{\alpha}{1 - \alpha \beta} c_{t-1} + \frac{1}{(1 - \beta)(1 - \alpha \beta)} x_t
    \]

    \[
= \frac{\alpha}{1 - \alpha \beta} c_{t-1} + \frac{1}{(1 - \beta)(1 - \alpha \beta)} x_t
\]

\end{enumerate}


\textit{From (2.3EX) and (2.4EX) we get}

\[
(1 + r) d_{t-1} = \sum_{j=0}^{\infty} \beta^j y_{t+j} - \sum_{j=0}^{\infty} \beta^j c_{t+j}
\]

\[
= \frac{1}{1 - \rho \beta} y_t - \frac{\alpha}{1 - \alpha \beta} c_{t-1} - \frac{1}{(1 - \beta)(1 - \alpha \beta)} (c_t - \alpha c_{t-1})
\]

\[
= \frac{1}{1 - \rho \beta} y_t + \frac{\alpha \beta}{(1 - \alpha \beta)(1 - \beta)} c_{t-1} - \frac{1}{(1 - \beta)(1 - \alpha \beta)} c_t
\]

\textit{So we have}

\[
\frac{dc_t}{dy_t} = \frac{(1 - \beta)(1 - \alpha \beta)}{1 - \rho \beta}
\]

\textbf{2.} For $d tb_t / dy_t$ to be negative, we need the above expression to be larger than unity. This requires

\[
\alpha < \frac{\rho - 1}{1 - \beta}
\]

So $\alpha$ must be negative. As $\rho \to 1$, $\alpha < 0$ is enough.


\end{document}
